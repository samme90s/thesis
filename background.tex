\section{Theoretical Background}
In many industries today, handling large quantities data is common. Text classification is one key application area where models are trained to classify data, or more specific, text documents into pre-defined classes. Traditionally, ML approaches have been used to build these classifiers. Early techniques depended on rule-based systems and simple algorithms like logistic regression or Naive Bayes, which relied on manually crafted features derived from structured data \cite{bing2011mining}. 

Data is generally divided into two main categories: structured and unstructured. \textbf{Structured data} is organized in a clear, predefined format. For example, consider an employee roster stored in a spreadsheet where each row represents an employee and columns indicate details such as name, employee ID, department, and phone number. Similarly, a financial report that lists company names, dates, and expense figures in a table is also structured data. Its orderly layout makes it easy for business users to read and analyze. \textbf{Unstructured data} lacks this uniform organization. Examples include emails, text documents, or social media posts, where information is presented in varying formats without a strict arrangement. Work orders are a typical example of unstructured data. A work order might include a lengthy description of maintenance tasks, lists of required materials, details on location and customer info, and other text that does not fit into neatly labeled columns \cite{ibm2023work}. Because each document can be formatted differently, unstructured data does not lend itself to simple, tabular representation \cite{ibm2025datadiff}.

However, the early techniques often fall short in capturing language nuances, context details, and the ambiguity in unstructured text. More recently, advances in natural language processing (NLP) have led to the development of large language models (LLMs) based on transformer architectures such as GPT and BERT. These models use self-attention to focus on important parts of the text and capture deep meaning by understanding relationships between words. There are even newer versions that add reasoning capabilities, which aim to improve decision-making in text classification. Learning from vast amounts of pre-trained data and use self-attention to track relationships between words, which significantly improves their ability to understand and predict next word context. For example, while traditional ML models focus on manually extracting features, modern LLMs such as the previously mentioned GPT and BERT can adapt to various types of text with minimal human intervention such as pre processing and structuring. Supporting this development, recent research has shown that LLMs -- especially when fine-tuned for particular domains -- exhibit not only advanced language understanding but also reasoning capabilities that can further improve text classification performance \cite{huang2024classification, andersson2024ikea, merritt2022transformer, nazyrova2024medical, wang2024classifiers}, capabilities central to the comparisons explored in this work (RQ1, RQ2, RQ3).

One of the most significant advantages of large language models over traditional approaches is their ability to generalize to new tasks using \textbf{zero-shot}, \textbf{one-shot}, and \textbf{few-shot} learning \cite{brown2020language}. These concepts refer to how much prior labeled data a model needs to perform a task. Zero-shot learning allows an LLM to classify text without being explicitly trained on any labeled examples. The model uses pre-trained knowledge to infer the most likely category based on a prompt describing the task. One-shot learning improves upon this by giving the model just one labeled example before making the classification. Finally, few-shot learning further enhances accuracy by providing multiple labeled examples in the prompt, allowing the model to generalize more effectively without requiring full retraining.

Switching focus to a machine learning technique such as recurrent neural networks or long short-term memory (LSTMs) shows capability in handling sequences of words and remembering long distances between related words. This makes it easier for the model to understand context without needing manual feature design. This technique was introduced in the late 90's and is far less complex and resource consuming than the latter transformer technique and is still considered a state-of-the-art algorithm \cite{wang2024classifiers, hochreiter1997long}. However, training and testing an RNN och LSTM model still requires a significant amount of resources when using a large dataset.

These datasets often require some processing involving actions such as bag of words (BoW) and term frequency-inverse document frequency (TF-IDF). Briefly explained, \cite{murel2024bagofwords} bag of words involves collecting the frequency of words in documents; more specifically, it's a feature extraction technique that models text data by creating an unstructured collection of all words in a document. The collection solely represents how often the words appear while ignoring grammar, word order, and context. Building upon this concept is TF-IDF, which can be described as a variation that further accounts for word frequency not just within one document, but across a whole corpus (a large collection of writings of a specific kind or on a specific subject), effectively giving more weight to terms that are significant to a specific document compared to common words found everywhere.

Although as Nazyrova et al. \cite{nazyrova2024medical} focuses on the medical application area within text classification, the principles are directly applicable to our context.  Traditionally, companies rely on manual processing of WOs, a practice that is both labor-intensive and error prone \cite{li2024work}.