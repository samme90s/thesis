\section{Data}

The data provided by UPTILT will be used to train and evaluate different models
for the purpose of this research. Any data presented will be dummy data and not representative
of the actual data provided by UPTILT. This applies to all features except those
that are necessary to understand the context, such as the type of work.

\subsection{Experiment setup}

The experiment is conducted using anonymized work orders, but may still contain sensitive
information. Therefore, when presenting potentially sensitive information, it
will be replaced as dummy data. This will be clarified beforehand.

\subsection{Dataset characteristics}

The data contains \colorbox{yellow}{UNKNOWN} work orders, without relations to each
other. It is unstructured, meaning that the data does not have predefined categories.
The data comes in the form of text (English) and numerical values. We have opted to
only use the titles and descriptions of the work orders, as these are the most
relevant for the classification task, and will be used to verify the correctness
of the LLMs.

\subsection{Data subset}

When working with the ML model, we will use a manually structured subset of the data.
Using a subset allows the primary focus to be on the LLMs, as manually labeling the
entire dataset would not be feasible within the time frame of this thesis.

\subsection{Data pre-processing}

\colorbox{yellow}{Explain how the data is pre-processed.}
% Explain how the data is pre-processed
% Example remove stop words, punctuation, etc.
% Tokenize for NLTK
% Include explaination of where our scripts are located.

% Use python script for this.

\subsection{Labeling strategy}

\colorbox{yellow}{Explain what labels we will use and why when structuring the
dataset.}
