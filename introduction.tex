\section{Introduction}

This section introduces the different aspects of the research and to provide the
main scope of the paper.
It will describe large language models (LLMs) and their applications in text
classification.
Since we are collaborating with UPTILT who are working on delivering a program
to customers within the service
business.
They aim to deliver a product meant to keep track of materials and hourly work
to automate the delivery of invoices and offers to the customer.
Due to the collaboration, we will mainly discuss importance of classifying
and in the context of work orders (WOs), but also provide a general conclusion
on the topic and contribute to the field of text classification and natural
language processing (NLP).

\subsection{Background}

In many industries today, working with and handling large quantities of
structured and unstructured data is extremely important.
One example is text classification which orders text based on their content to a
specific label.

% Lite konstig mening
Traditionally, ML methods such as Support Vector Machines (SVM), Random Forest
and Naive Bayes classifiers have been used for text classification tasks,
and automating this process can be done by using one of the former models.

Training a model means analyzing the content of the text and identifying
features that are relevant, so computer systems can, for example, use this
technique for identifying spam and filtering documents \cite{dalal2011}.
Compared to the former approaches that requires manual training and configuration,
this paper aims to find a more effective solution when dealing with text classification
and exploring the usage of LLM in relation with WOs.

\subsubsection{Large language models}

LLMs are a subgroup of AI that work on the knowledge of next word context and are often utilized
in the context of allowing the user to prompt it with instructions or questions.
LLMs utilize pre-trained knowledge, unlike traditional ML models, which require manual training,
balanced datasets and human interaction.
LLMs can analyze data and recognize patterns with minimal human interaction.
This makes LLMs highly effective in applications like (NLP) \cite{andersson2024},
and in our case classification of WOs.

\subsubsection{Text classification of work orders}

WOs is a document that outlines the details of a maintenance task.
This data comes in the form of text and can through proper utilization be the key component when categorizing using LLMs.
WOs most commonly contains information regarding the material required to complete such task,
but also the geographic location, customer etc.
Today, many companies still rely on manual processing, which often leads to mistakes.
Throughout our work we will focus on providing a better solution than the current manual processing,
by using LLMs \cite{ibm2023} \cite{li2024}.

Assigning text to pre-defined categories is the key of text classification,
sometimes named categorization, but we will stick to the former.
These may include areas such as topic labeling or news classification.
It has been further proven that the latest LLMs currently posses great generative capabilities,
along with understanding and reasoning within language(s) as previously mentioned.
They therefore have become more efficient than humans when it comes to NLP tasks \cite{zhang2024}.

\subsubsection{Related Work}

Huang and He \cite{huang2024} explain the cost of manual labor when it comes to tagging and categorizing text.
Furthermore, the fine-tuning and choice of parameters can influence the final behavior of the trained ML model.
They explain how the latest LLMs have showcased "remarkable reasoning performance across a wide range of
NLP tasks".
We aim to follow this trail and further analyze the efficiency and performance of categorizing text within the
context of WOs.

\bigskip
Recent advances in LLMs have shown that they can be fine-tuned to effectively classify structured data.
For example, GPT-4, LLaMA 2, and ChatGLM 2 have shown remarkable performance in text classification \cite{zhang2024}.
We will be discussing and evaluating the efficiency and accuracy of using LLMs
to classify structured WOs, unlike previous studies, which focuses more on general NLP tasks.
Our paper gives an insight into whether LLMs can be used with correct prompting to classify WOs.

\bigskip
Current research has demonstrated that transformer-based models
(a neural network architecture, meaning it learns by tracking relationships between words \cite{merritt2022}),
such as BERT and other domain-specific variants, significantly improve text classification processes through
transfer learning and fine-tuning \cite{nazyrova2024}.
While Nazyrova et al. focuses on medical text classification and our focus lies within WO classification,
their findings on the topic of fine-tuning LLMs is relevant for our paper and its research.
