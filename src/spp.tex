% Fundamental framework for your document structure
\documentclass{article}
% Needed for your colored text boxes in the elevator pitch
% (\textcolor{green}{...} etc.)
\usepackage{xcolor}
% Required for PDF hyperlinks and proper PDF metadata
% (especially important when combined with natbib)
\usepackage{hyperref}
% Crucial for your citation management 
% (\cite commands) and bibliography style
\usepackage[numbers,sort&compress]{natbib}

\title{Student Project Proposal}
\author{Samuel Svensson \and Hampus Tuisku}
\date{\today}

\begin{document}
\maketitle

\section{Pre-Introduction}

\subsection{LNU Supervisor}
Oxana Lundström

\subsection{Supervision status according to student}

\begin{itemize}
      \item [ ] Currently working with (ongoing project)
      \item [ ] Agreed, but not started
      \item [x] Wish to work with (or unclear of status)
      \item [ ] No preference, help me find one
\end{itemize}

\subsection{Cooperative Partners}

We will cooperate with UPTILT, who are currently working on delivering a program
to customers within the service business.
They aim to deliver a product meant to keep track of materials and hourly work
to automate the delivery of invoices and offers to the customer.

We have continous discussions and meetings with the company.

\subsection{Preliminary Title}

Automated Text Categorization for Work Orders using LLM

\subsection{Elevator Pitch}

\textcolor{green}{
      Automatic text categorization is an important step when analyzing textual data,
      and the approach of Large Language Models (LLMs) have emerged as an alternative
      to the traditional Machine Learning (ML).
}
\textcolor{orange}{
      ML often requires balanced, labeled datasets for effective training
      and demands significant manual effort, in contrast to LLMs that
      leverages pre-trained contextual knowledge.
}
\textcolor{blue}{
      We aim to use LLMs to categorize work order (WO) titles and descriptions
      through a rigorous method,
}
\textcolor{red}{
      and evaluate the impact of different prompts and strategies to find
      an effective solution to the subject at hand.
}

\subsection{Steps/Milestones/Actions}

\begin{enumerate}
      \item Find related work.
      \item Create two questions related to our different strategies used
            when using the LLM (this is for each of us to have an area of
            greater responsibility).
      \item Define method (be creative here to find interresting approaches
            to our problem).
      \item Conduct experiment.
      \item Analyze results.
\end{enumerate}

\subsection{Risks}

\begin{itemize}
      \item No previous experience in creating a method, so we will have to rely on our
            supervisor here.
      \item Finding creative and concrete ways to test the effectiveness and accuracy.
\end{itemize}

\section{Introduction}

\subsection{Background}

% Description of the background of the research and the application areas of your work. 

% Define both application area and research area. Think funnel so go from broad to specific
% (as discussed during Workshop 1).
% Application areas = areas outside of CS that CS is applied to.
% Research area = area within CS that you do a knowledge contribution to.


% Use references especially on the CS parts.

% Describe the current knowledge or state and describe why a change or new knowledge is needed.
% Motivate from a societal OR economic OR ethical points of view. 

% Think of what is the target group for your thesis? Try to make this as wide as possible and this
% target group must be within computer science community. Also make sure it is beyond a specific target
% ( such as a company).

% If you are going to develop something (eg. prototype, app, web app, …) then describe the closest solutions.
% If there are many have a table. You can show the important features.

In many industries today, working with and handling large quantities of structured and unstructured data is extremely
important. One example is categorization of work orders in fields such as manufacturing, logistics and maintenance.
\bigskip

Traditionally, machine learning methods such as Support Vector Machines (SVM), Random Forest and Naïve Bayes classifiers
have been used for text classification task. These machine learning methods have also been applied to the topic of
this paper, i.e., text classification. These approaches require manual testing and overlooking. This can result in
problems when it comes to scalability and handling different text formats.
\bigskip

To address the question if we can't find a more effective solution to this current workflow when dealing with
text classification, this paper explores the usage of Large Language Models (LLMs) for automated text categorization
for work orders.

\subsubsection{Work Orders}

Work orders is a document that outlines the details of a maintenance task or the material and more required
to complete such task \cite{ibm2023}.

\colorbox{yellow}{add more here...}

\subsubsection{Automated Text Categorization}

LLMs or often described as AI nowadays and allows the user to prompt it with instructions or questions.

\colorbox{yellow}{continue with a short description of LLM and add source/reference}

Assigning text to pre-defined categories is the key of text categorization
(sometimes named classification, but we will stick to the former).
These may include areas such as topic labeling or news categorization \cite{zhang2024}.
It has further proved that the latest LLMs currently posses great generative capabilities, understanding
and reasoning within language(s).
They therefore have become more efficient than humans when it comes to NLP (Natural Language Processing)
tasks.

\colorbox{yellow}{continue with the explaination of text categorization}

\subsubsection{Related Work}
\colorbox{yellow}{Reveal other studies and their conclusions, position ourselves and motivate...}

% Position yourself in a research area within Computer Science according to instructions given during workshop.
% So in this section you only focus on CS!!! 

% Minimum two articles published in CS conferences or journals.
% However do make sure you find the important and most relevant works and that it is enough to motivate a
% knowledge gap or that your problem is a CS problem.

% Summarize what others have done as well as not have done with one or two sentences.
% Summarize their conclusion with one or two sentences.
% Then finally position yourself in relation to these related works eg.
% “this is close to what I intend to do”, “I build on top of this”,
% “my work is different from this” as discussed in Workshop 1 and 2. 

% Again make sure this is within CS and on topic of your research area.
% Check the conference or journal so that it is a scientific CS venue. 

% Is there active research in this area? Answer this question by looking at when the papers were published. 

Huang and He \cite{huang2024} explain the cost of manual labor when it comes to tagging and categorizing text.
Furthermore, the fine-tuning and choice of parameters can influence the final behaviour of the trained
ML model.
They explain how the latest LLMs have showcased "remarkable reasoning performance across a wide range of
NLP tasks".
We aim to follow this trail and further analyze the effeciency and performance of categoring text or as in our
case work orders.
\bigskip

Advances have been made recently in Large Language Models (LLMs) when it comes to Natural Language Processing (NLP) tasks,
such as text classification.
GPT-4, LLaMA 2, and ChatGLM 2 have shown remarkable performance in various NLP-tasks, including text classification.
\cite{zhang2024}.
Our research in this paper will be to discuss and evaluate the efficiency and accuracy of using Large Language Models
to text classify structured work orders. Unlike previous studies, which focus more on general NLP tasks,
our paper give an insight into whether LLMs can be fine-tuned to effectively classify structured work orders and compare their
performance to traditional text classification methods.


\bibliographystyle{IEEEtran}
\bibliography{references}

\end{document}
