\documentclass{article}
\usepackage[utf8]{inputenc}
\usepackage{xcolor}
\usepackage{hyperref}

\title{Student Project Proposal}
\author{Hampus Tuisku \and Samuel Svensson}
\date{\today}

\begin{document}
\maketitle

\section{Pre-Introduction}

\subsection{LNU Supervisor}
Oxana Lundström

\subsection{Supervision status according to student}

\begin{itemize}
      \item [ ] Currently working with (ongoing project)
      \item [ ] Agreed, but not started
      \item [x] Wish to work with (or unclear of status)
      \item [ ] No preference, help me find one
\end{itemize}

\subsection{Cooperative Partners}

We will cooperate with UPTILT, who are currently working on delivering a program
to customers within the service business.
They aim to deliver a product meant to keep track of materials and hourly work
to automate the delivery of invoices and offers to the customer.

We have continous discussions and meetings with the company.

\subsection{Preliminary Title}

Automated Text Categorization for Work Orders using LLM

\subsection{Elevator Pitch}

\textcolor{green}{
      Automatic text categorization is an important step when analyzing textual data,
      and the approach of Large Language Models (LLMs) have emerged as an alternative
      to the traditional Machine Learning (ML).
}
\textcolor{orange}{
      ML often requires balanced, labeled datasets for effective training
      and demands significant manual effort, in contrast to LLMs that
      leverages pre-trained contextual knowledge.
}
\textcolor{blue}{
      We aim to use LLMs to categorize work order (WO) titles and descriptions
      through a rigorous method,
}
\textcolor{red}{
      and evaluate the impact of different prompts and strategies to find
      an effective solution to the subject at hand.
}

\subsection{Steps/Milestones/Actions}

\begin{enumerate}
      \item Find related work.
      \item Create two questions related to our different strategies used
            when using the LLM (this is for each of us to have an area of
            greater responsibility).
      \item Define method (be creative here to find interresting approaches
            to our problem).
      \item Conduct experiment.
      \item Analyze results.
\end{enumerate}

\subsection{Risks}

\begin{itemize}
      \item No previous experience in creating a method, so we will have to rely on our
            supervisor here.
      \item Finding creative and concrete ways to test the effectiveness and accuracy.
\end{itemize}

\section{Introduction}

\subsection{Background}

\subsubsection{Text Categorization}
\colorbox{yellow}{Explain the area...}

\subsubsection{Related Work}
\colorbox{yellow}{Reveal other studies and their conclusions, position ourselves and motivate...}

Huang and He [1] explain the cost of manual labor when it comes to tagging and categorizing text.
Furthermore, the fine-tuning and choice of parameters can influence the final behaviour of the trained
ML model.
They explain how the latest LLMs have showcased "remarkable reasoning performance across a wide range of
NLP tasks".
We aim to follow this trail and further analyze the effeciency and performance of categoring text or as in our
case work orders.

\section{References}

 [1]
C. Huang and G. He, "Text Clustering as Classification with LLMs,"
arXiv preprint arXiv:2410.00927, Sep. 2024, revised Jan. 2025. [Online].
Available: \url{https://doi.org/10.48550/arXiv.2410.00927}

\end{document}
